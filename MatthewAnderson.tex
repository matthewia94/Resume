%% The MIT License (MIT)
%%
%% Copyright (c) 2015 Daniil Belyakov
%%
%% Permission is hereby granted, free of charge, to any person obtaining a copy
%% of this software and associated documentation files (the "Software"), to deal
%% in the Software without restriction, including without limitation the rights
%% to use, copy, modify, merge, publish, distribute, sublicense, and/or sell
%% copies of the Software, and to permit persons to whom the Software is
%% furnished to do so, subject to the following conditions:
%%
%% The above copyright notice and this permission notice shall be included in all
%% copies or substantial portions of the Software.
%%
%% THE SOFTWARE IS PROVIDED "AS IS", WITHOUT WARRANTY OF ANY KIND, EXPRESS OR
%% IMPLIED, INCLUDING BUT NOT LIMITED TO THE WARRANTIES OF MERCHANTABILITY,
%% FITNESS FOR A PARTICULAR PURPOSE AND NONINFRINGEMENT. IN NO EVENT SHALL THE
%% AUTHORS OR COPYRIGHT HOLDERS BE LIABLE FOR ANY CLAIM, DAMAGES OR OTHER
%% LIABILITY, WHETHER IN AN ACTION OF CONTRACT, TORT OR OTHERWISE, ARISING FROM,
%% OUT OF OR IN CONNECTION WITH THE SOFTWARE OR THE USE OR OTHER DEALINGS IN THE
%% SOFTWARE.

% !TeX program = xelatex

% The font could be set to Windows-specific Calibri by using the 'calibri' option
\documentclass[]{mcdowellcv}
\renewcommand{\familydefault}{\sfdefault}

% For mathematical symbols
\usepackage{amsmath}
\usepackage[
    colorlinks=true
]{hyperref}
\usepackage{xcolor}

% For better comments
\usepackage{comment}
\includecomment{cruise}
\includecomment{boeing}
\excludecomment{aigrader}
\excludecomment{johndeere}
\excludecomment{rockmech}
\excludecomment{garmin}
\excludecomment{vijon}
\excludecomment{technical-experience}

\AtBeginDocument{
  \hypersetup{
    linkcolor=blue,   
    urlcolor=blue}
}
\begin{document}

% Set applicant's personal data for header
\name{Matthew Anderson}
\address{Seattle, WA \linebreak \href{https://www.linkedin.com/in/matthewianderson/}{linkedin.com/in/matthewianderson}}
\contacts{(314)-210-8316 \linebreak matthewia94@gmail.com}

% Header
\makeheader

% Employment
\begin{cvsection}{Employment}

% Cruise LLC
\begin{cruise}
    \begin{cvsubsection}{Senior Machine Learning Engineer II - Perception}{Cruise LLC}{Jun 2018 - Present}
        \begin{itemize}
            \item Implemented, tested, analyzed, deployed, and monitored safety critical features for improving overall perception performance:
            \begin{itemize}
                \item Added support for object velocity prediction in an early fusion detection model reducing false positive motion predictions by over 100\%
                \item Iterated on data and loss weighting to recall out of distribution objects traveling at extreme speeds
                \item Experimented with new architectures for perception such as TrackFormer and DSVT
                \item Identified and planned solutions for observed road problems including: harsh braking around cones, poor classification of large animals, sensor specific ghosting artifacts
            \end{itemize}
            \item Acted as a technical leader on the team through mentorship and project management:
            \begin{itemize}
                \item Led, reviewed, and approved model release analyses across perception to ensure safe deployments
                \item Ran experiment reviews and led discussions to share learnings and align on follow-up experiments
                \item Worked with partner teams to improve developer velocity and safety through automations in analysis
                \item Documented and formalized processes to ensure a consistently high safety bar within perception
            \end{itemize}
            \item Responsible for maintaining the onboard vehicle stack to extract model input features and process model outputs for downstream consumption
            \item Automated data pipelines with internal libraries to fully automate model data generation, retrain, and deployment with a single command; Increased developer iteration speed by 3x and enabled the team to merge model iterations on a monthly cadence when necessary
            \item Collaborated to replace legacy tracking systems with machine learning solutions 
        \end{itemize}
    \end{cvsubsection}
\end{cruise}

% Boeing
\begin{boeing}
    \begin{cvsubsection}{Software Engineer - Sensors and Special Programs}{The Boeing Company}{Jul 2016 - May 2018}
        \begin{itemize}
            \item Implemented sensor data fusion algorithms with a small team of 5 contributors
            \item Implemented and maintained an application to perform offline simulations which integrated with internally developed tools to evaluate algorithm performance
            \item Prototyped, tested and analyzed a new Kalman filtering approach to increase performance of algorithms and enable new capabilities
        \end{itemize}
    \end{cvsubsection}
\end{boeing}
    
% MS&T AI Grader
\begin{aigrader}
    \begin{cvsubsection}[2]{Teaching Assistant}{Missouri University of Science and Technology}{Jan 2016 - May 2016}
        \begin{itemize}
            \item Assisted students with troubleshooting of programming assignments in Introduction to Artificial Intelligence, focusing on adversarial search algorithms
            \item Designed and tested programming problems to provide interesting and challenging assignments which could be solved using increasingly sophisticated search algorithms
        \end{itemize}
    \end{cvsubsection}
\end{aigrader}

% John Deere
\begin{johndeere}
    \begin{cvsubsection}{Product Engineering Intern}{John Deere}{May 2015 - Aug 2015}
        \begin{itemize}
            \item Researched processes for streamlining mobile application localization
            \item Implemented commercial database system for reuse of string translations on a mobile platform team
            \item Developed scripts to prepare localization files for translation
        \end{itemize}
    \end{cvsubsection}
\end{johndeere}
    
% MS&T Rock Mechanics
\begin{rockmech}
    \begin{cvsubsection}[2]{Student Technical Assistant}{Missouri University of Science and Technology}{Oct 2013 - Nov 2014}        
        \begin{itemize}
            \item Maintained program to remove foliage from LIDAR point cloud to unmask objects
            \item Optimized program utilizing CUDA for parallel computing, reducing application runtime by 80\%
            \item Packaged LIDAR data processing solution for easy integration into ROS based robotics platforms
        \end{itemize}
    \end{cvsubsection}
\end{rockmech}
    
% Garmin
\begin{garmin}
    \begin{cvsubsection}{Software Engineering Intern}{Garmin International}{May 2014 - Aug 2014}   
        \begin{itemize}
            \item Designed and implemented an automated software system to perform hardware regression tests
            \item Integrated testing system with standardized error logging application through inter-team collaboration
        \end{itemize}
    \end{cvsubsection}
\end{garmin}

% Vi-Jon
\begin{vijon}
    \begin{cvsubsection}{Information Systems Intern}{Vi-Jon, Inc.}{May 2013 - Aug 2013}
        \begin{itemize}
            \item Developed an application to track supplies within the department
            \item Implemented automatic inventory reporting to streamline the process of ordering supplies
        \end{itemize}
    \end{cvsubsection}
\end{vijon}

\end{cvsection}

% Education
\begin{cvsection}{Education}
    \begin{cvsubsection}[2]{Rolla, MO}{Missouri University of Science and Technology}{}
        \begin{itemize}
            \item B.S. in Computer Science, Magna Cum Laude
            \item B.S. in Computer Engineering, Magna Cum Laude
        \end{itemize}
    \end{cvsubsection}
\end{cvsection} 

% Experience and Awards
\begin{technical-experience}
    \begin{cvsection}{Technical Experience}
        \begin{cvsubsection}{}{}{}  
            \begin{itemize} 
                \item \textbf{Competition Robotics Team (2012 - 2016)}: Led the programming division and contributed to the development of a fully autonomous robot designed to compete in the Intelligent Ground Vehicle Competition (IGVC). Focused on path finding and computer vision algorithms for navigating an outdoor obstacle course utilizing LIDAR and stereo optics.
            \end{itemize}
        \end{cvsubsection}
    \end{cvsection}
\end{technical-experience}

% Languages and Tech
\begin{cvsection}{Languages and Technologies}
    \begin{cvsubsection}{}{}{}  
        \begin{itemize}
            \item C++; Python; PyTorch; BigQuery; Robot Operating System (ROS); Bazel; CMake; Git; Boost;
        \end{itemize}
    \end{cvsubsection}
\end{cvsection}

\end{document}
